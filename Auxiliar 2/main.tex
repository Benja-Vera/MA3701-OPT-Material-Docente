\documentclass{article}
\input{imports.tex}
\input{config.tex}

\begin{document}
\noindent \textbf{MA3701 Optimización}\\
\textbf{Profesor:} Alejandro Jofré\\
\textbf{Auxiliar:} Benjamín Vera Vera


\begin{center}
    \Huge{\textbf{Auxiliar 2}}\\
\textit{\large{Optimización sin restricciones}}\\
    \normalsize
    13 de agosto de 2025
\end{center}

\begin{enumerate}
	\item Sea $A \in \mathbb{R}^{m \times n}$. Pruebe que $A^\top A \in \mathbb{R}^{n \times n}$ es semidefinida positiva. Entregue condiciones para que sea definida positiva.

	\item \textbf{(Regresión de mínimos cuadrados lineal
)} Suponga que se cuenta con $N$ datos $(x_i, y_i)_{i=1}^N$ en que $x_i \in \mathbb{R}^d$ representa *características* o *atributos* de un objeto que se piensa que predicen la *respuesta* $y_i$. A modo de ejemplo, se puede pensar que $x_i$ representa diferentes datos asociados a un terreno $i$ --tales como tamaño, distancia a la estación de metro más cercana, etc-- que se buscan relacionar con el precio $y_i$ del mismo. Se busca entonces modelar esta relación como una función lineal afín de tipo
$$
	y = a^\top x + b
$$
con $a \in \mathbb{R}^d, b \in \mathbb{R}$ escogidos adecuadamente. Estudiaremos el problema de encontrar $a, b$ bajo el criterio de minimizar el *error cuadrático medio* tomado sobre los datos $x_i, y_i$ que se tiene. Es decir, se busca encontrar $a, b$ que minimicen la función
$$
	f(a, b) = \sum_{i=1}^N (y_i - (a^\top x + b))^2.
$$
A modo de simplificación, dado $x_i \in \mathbb{R}^d$, se puede añadir una coordenada adicional definida como $x_{i, d+1} = 1$ y definir también $\theta = (a, b)$ de modo que tenemos
$$
	f(\theta) = \sum_{i=1}^N (y_i - \theta^\top x_i)^2.
$$

\begin{enumerate}
	\item Pruebe que la función $f(\theta)$ es convexa.
	\item Asumiendo que la matriz $X = (x_{i,j})_{i=1, j = 1}^{N, d+1}$ tiene columnas linealmente independientes, encuentre el único minimizador $\theta^*$ para el problema de mínimos cuadrados.
	\item Deduzca la fórmula explícita para $(a, b)$ en el caso particular $d=1$.
\end{enumerate}

\end{enumerate}


\end{document}
