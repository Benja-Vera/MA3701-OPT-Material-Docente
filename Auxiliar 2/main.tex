\documentclass{article}
\usepackage[letterpaper, rmargin=3em, lmargin=3em, textheight=63em]{geometry}
\usepackage{fancyhdr}
\usepackage[spanish]{babel}
\usepackage[dvipsnames]{xcolor}
\usepackage{graphicx}
\usepackage{wrapfig}
\usepackage{setspace}
\usepackage{hyperref}

\usepackage{amssymb}
\usepackage{amsmath}

\usepackage{charter}
\usepackage{physics}

\usepackage{multicol}
\usepackage{tikz}
% GEOMETRY
\setlength{\parskip}{1em}
\pagestyle{fancy}
\lhead{Facultad de Ciencias Físicas y Matemáticas}
\rhead{Universidad de Chile}
\cfoot{ }

\renewcommand{\labelenumi}{\normalsize\bfseries P\arabic{enumi}.}
%\renewcommand{\labelenumii}{\normalsize\bfseries (\alph{enumii})}
\renewcommand{\labelenumiii}{\normalsize\bfseries \roman{enumiii})}

% Alfabeto
\newcommand{\A}{\mathcal{A}}
\newcommand{\B}{\mathcal{B}}
\newcommand{\C}{\mathcal{C}}
\newcommand{\E}{\mathcal{E}}
\newcommand{\F}{\mathcal{F}}
\newcommand{\I}{\mathcal{I}}
\newcommand{\K}{\mathcal{K}}
\renewcommand{\L}{\mathcal{L}}
\newcommand{\M}{\mathcal{M}}
\newcommand{\N}{\mathbb{N}}
\renewcommand{\P}{\mathcal{P}}
\newcommand{\Q}{\mathbb{Q}}
\newcommand{\R}{\mathbb{R}}
\renewcommand{\S}{\mathcal{S}}
\newcommand{\T}{\mathcal{T}}
\newcommand{\Z}{\mathbb{Z}}

\DeclareMathOperator{\sen}{sen}
\DeclareMathOperator{\senh}{senh}
\DeclareMathOperator{\tg}{tg}
\DeclareMathOperator{\dom}{dom}
\DeclareMathOperator{\dist}{dist}
\DeclareMathOperator*{\argmin}{argmin}
\DeclareMathOperator{\arccosh}{arccosh}

\renewcommand{\epsilon}{\varepsilon}
\renewcommand{\phi}{\varphi}
\newcommand{\dprod}[2]{\langle #1 , #2 \rangle}
\newcommand{\prox}{\mathbf{prox}}

\hypersetup{
    colorlinks=true,
    linkcolor=blue,
    filecolor=magenta,      
    urlcolor=blue,
}

\begin{document}
\noindent \textbf{MA3701 Optimización}\\
\textbf{Profesor:} Alejandro Jofré\\
\textbf{Auxiliar:} Benjamín Vera Vera


\begin{center}
    \Huge{\textbf{Auxiliar 2}}\\
\textit{\large{Optimización sin restricciones}}\\
    \normalsize
    13 de agosto de 2025
\end{center}

\begin{enumerate}
	\item Sea $A \in \mathbb{R}^{m \times n}$. Pruebe que $A^\top A \in \mathbb{R}^{n \times n}$ es semidefinida positiva. Entregue condiciones para que sea definida positiva.

	\item \textbf{(Regresión de mínimos cuadrados lineal
)} Suponga que se cuenta con $N$ datos $(x_i, y_i)_{i=1}^N$ en que $x_i \in \mathbb{R}^d$ representa *características* o *atributos* de un objeto que se piensa que predicen la *respuesta* $y_i$. A modo de ejemplo, se puede pensar que $x_i$ representa diferentes datos asociados a un terreno $i$ --tales como tamaño, distancia a la estación de metro más cercana, etc-- que se buscan relacionar con el precio $y_i$ del mismo. Se busca entonces modelar esta relación como una función lineal afín de tipo
$$
	y = a^\top x + b
$$
con $a \in \mathbb{R}^d, b \in \mathbb{R}$ escogidos adecuadamente. Estudiaremos el problema de encontrar $a, b$ bajo el criterio de minimizar el *error cuadrático medio* tomado sobre los datos $x_i, y_i$ que se tiene. Es decir, se busca encontrar $a, b$ que minimicen la función
$$
	f(a, b) = \sum_{i=1}^N (y_i - (a^\top x + b))^2.
$$
A modo de simplificación, dado $x_i \in \mathbb{R}^d$, se puede añadir una coordenada adicional definida como $x_{i, d+1} = 1$ y definir también $\theta = (a, b)$ de modo que tenemos
$$
	f(\theta) = \sum_{i=1}^N (y_i - \theta^\top x_i)^2.
$$

\begin{enumerate}
	\item Pruebe que la función $f(\theta)$ es convexa.
	\item Asumiendo que la matriz $X = (x_{i,j})_{i=1, j = 1}^{N, d+1}$ tiene columnas linealmente independientes, encuentre el único minimizador $\theta^*$ para el problema de mínimos cuadrados.
	\item Deduzca la fórmula explícita para $(a, b)$ en el caso particular $d=1$.
\end{enumerate}

\end{enumerate}


\end{document}
