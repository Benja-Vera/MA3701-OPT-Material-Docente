\documentclass{article}
\usepackage[letterpaper, rmargin=3em, lmargin=3em, textheight=63em]{geometry}
\usepackage{fancyhdr}
\usepackage[spanish]{babel}
\usepackage[dvipsnames]{xcolor}
\usepackage{graphicx}
\usepackage{wrapfig}
\usepackage{setspace}
\usepackage{hyperref}

\usepackage{amssymb}
\usepackage{amsmath}

\usepackage{charter}
\usepackage{physics}

\usepackage{multicol}
\usepackage{tikz}
% GEOMETRY
\setlength{\parskip}{1em}
\pagestyle{fancy}
\lhead{Facultad de Ciencias Físicas y Matemáticas}
\rhead{Universidad de Chile}
\cfoot{ }

\renewcommand{\labelenumi}{\normalsize\bfseries P\arabic{enumi}.}
%\renewcommand{\labelenumii}{\normalsize\bfseries (\alph{enumii})}
\renewcommand{\labelenumiii}{\normalsize\bfseries \roman{enumiii})}

% Alfabeto
\newcommand{\A}{\mathcal{A}}
\newcommand{\B}{\mathcal{B}}
\newcommand{\C}{\mathcal{C}}
\newcommand{\E}{\mathcal{E}}
\newcommand{\F}{\mathcal{F}}
\newcommand{\I}{\mathcal{I}}
\newcommand{\K}{\mathcal{K}}
\renewcommand{\L}{\mathcal{L}}
\newcommand{\M}{\mathcal{M}}
\newcommand{\N}{\mathbb{N}}
\renewcommand{\P}{\mathcal{P}}
\newcommand{\Q}{\mathbb{Q}}
\newcommand{\R}{\mathbb{R}}
\renewcommand{\S}{\mathcal{S}}
\newcommand{\T}{\mathcal{T}}
\newcommand{\Z}{\mathbb{Z}}

\DeclareMathOperator{\sen}{sen}
\DeclareMathOperator{\senh}{senh}
\DeclareMathOperator{\tg}{tg}
\DeclareMathOperator{\dom}{dom}
\DeclareMathOperator{\dist}{dist}
\DeclareMathOperator*{\argmin}{argmin}
\DeclareMathOperator{\arccosh}{arccosh}

\renewcommand{\epsilon}{\varepsilon}
\renewcommand{\phi}{\varphi}
\newcommand{\dprod}[2]{\langle #1 , #2 \rangle}
\newcommand{\prox}{\mathbf{prox}}

\hypersetup{
    colorlinks=true,
    linkcolor=blue,
    filecolor=magenta,      
    urlcolor=blue,
}

\begin{document}
\noindent \textbf{MA3701 Optimización}\\
\textbf{Profesor:} Alejandro Jofré\\
\textbf{Auxiliar:} Benjamín Vera Vera


\begin{center}
    \Huge{\textbf{Control 1}}\\
\textit{\large{Tiempo: 3:00}}\\
    \normalsize
    8 de septiembre de 2025
\end{center}

\begin{enumerate}
	\item Suponga que se dispone de \(n\) fábricas donde se produce cereal, así como \(m\) tiendas que lo comercializan. Cada fábrica \(i\) puede producir como máximo \(s_i\) kilogramos de cereal, con un costo de \(b_i\) por kilogramo, mientras que cada tienda \(j\) tiene una demanda fija de \(d_j\) kilogramos que debe satisfacerse. El costo de transportar un kilogramo de cereal desde la fábrica \(i\) hasta la tienda \(j\) es \(c_{ij}\). Suponga durante este problema que la demanda agregada de todas las tiendas \(\sum_{j=1}^m d_j\) es menor o igual que la capacidad de producción agregada \(\sum_{i=1}^n s_i\). Formule un modelo de optimización que permita decidir simultáneamente cuántos kilogramos producir en cada fábrica y cuántos enviar desde cada fábrica \(i\) hacia cada tienda \(j\), de manera de minimizar el costo total de producción y transporte. En su formulación, defina claramente las variables de decisión, y explique el significado de cada restricción y de la función objetivo.

		\textbf{Solución:} Consideremos las siguientes variables de decisión:
		\begin{itemize}
			\item \(y_i\) : kilogramos producidos en la fábrica \(i\).
			\item \(x_{ij}\) : kilogramos a enviar desde la fábrica \(i\) a la tienda \(j\).
		\end{itemize}
		\[
		\text{\textcolor{ForestGreen}{(1.0) nombrar variables.}}
		\]
		De este modo, el costo total de producción y transporte viene dado por
		\[
			f(x, y) = \sum_{i=1}^n b_i y_i + \sum_{i=1}^n \sum_{j=1}^m c_{ij} x_{ij}.
		\]
		\[
			\text{\textcolor{ForestGreen}{(1.0) función objetivo.}}
		\]
		Las restricciones propias de este problema quedan entonces dadas por:
		\begin{itemize}
			\item \textbf{Positividad}
				\begin{align*}
					\forall i = 1, \dots, n &: y_i \geq 0, \\
					\forall i = 1, \dots, n, j = 1, \dots, m &: x_{ij} \geq 0.
				\end{align*}
			\[
				\text{\textcolor{ForestGreen}{(0.5) variables deben ser positivas.}}
			\]
		\item \textbf{Demanda}
			\[
				\forall j = 1, \dots, m : \sum_{i=1}^n x_{ij} \geq d_j.
			\]
			\[
				\text{\textcolor{ForestGreen}{(1.0) restricción de demanda (NOTA: también es válido que la restricción sea de igualdad).}}
			\]
		\item \textbf{Capacidad}
			\[
				\forall i = 1, \dots, n : y_i \leq s_i.
			\]
			\[
				\text{\textcolor{ForestGreen}{(1.0) restricción de producción máxima.}}
			\]
		\item \textbf{Envío de producción}
			\[
				\forall i = 1, \dots, n : y_i = \sum_{j=1}^m x_{ij}.
			\]
			\[
				\text{\textcolor{ForestGreen}{(1.0) relación entre las dos variables (NOTA: la restricción también puede ser en el sentido \(\geq\)).}}
			\]
		\end{itemize}
		En conclusión, el problema de optimización queda como sigue:
		\begin{align*}
			\min_{x,y} & \left\{ \sum_{i=1}^n b_i y_i + \sum_{i=1}^n \sum_{j=1}^m c_{ij}x_{ij}\right\} \\
				   & \forall j : \sum_{i=1}^n x_{ij} \geq d_j \\
				   & \forall i : y_i = \sum_{j=1}^m x_{ij} \\
				   & \forall i, j : x_{ij} \geq 0 \\
				   & \forall i : 0 \leq y_i \leq s_i.
		\end{align*}
		\[
			\text{\textcolor{ForestGreen}{(0.5) concluir.}}
		\]
		\textcolor{ForestGreen}{\textit{Solución alternativa:} Notar que la ecuación \(y_i = \sum_{j} x_{ij} \) permite eliminar la variable \(y_i\). Es posible plantear el problema solo en términos de los \(x_{ij}\).}
	\item Considere el siguiente problema de minimización:
		\begin{align*}
			\min_{x, y} \; & x^2 + y^2 \\
				       & \frac{x^2}{4} + y^2 = 1.
		\end{align*}
		\begin{enumerate}
			\item (1 pt) Obtenga las condiciones de KKT para este problema, utilícelas para obtener todos los puntos críticos con sus multiplicadores asociados.
			\item (2 pt) Para los puntos \((x_0, y_0)\) obtenidos en a), obtenga el cono de direcciones críticas \(K(x_0, y_0)\).
			\item (1 pt) Obtenga la matriz Hessiana del Lagrangeano \(\nabla_{x, y}^2 L(x_0, y_0, \mu_0)\) en los puntos obtenidos.
			\item (2 pt) Utilizando la información anterior, decida cuáles de estos puntos son mínimos locales del problema. Concluya entregando un candidato a mínimo global del problema.
		\end{enumerate}

		\textbf{Solución:} 
		\begin{enumerate}
			\item Las condiciones de KKT quedan
		\begin{align*}
			2x + \mu \cdot \frac{x}{2} &= 0, \\
			2y + \mu \cdot 2y &= 0, \\
			\frac{x^2}{4} + y^2 &= 1.
		\end{align*}
		\[
			\text{\textcolor{ForestGreen}{(0.5) obtener condiciones.}}
		\]
		Las cuales factorizando se pueden escribir como:
		\begin{align}
			x \left(2 + \frac{\mu}{2}\right) &= 0, \label{KKT1} \\
			y (2 + 2 \mu) &= 0, \label{KKT2} \\
			\frac{x^2}{4} + y^2 &= 1. \label{KKT3}
		\end{align}
		Ahora, para resolver, consideramos casos:
		\begin{itemize}
			\item Si \(x = 0\), por \eqref{KKT3}, \(y = \pm 1\). En \eqref{KKT2}, \(\mu = -1\), obteniendo
				\[
					(x, y, \mu) = (0, \pm 1, -1).
				\]
			\item Si \(x \neq 0\), por \eqref{KKT1}, \(\mu = -4\). En \eqref{KKT2}, \(y = 0\). En \eqref{KKT3}, \(x = \pm 2\), obteniendo
				\[
					(x, y, \mu) = (\pm 2, 0, -4).
				\]
		\end{itemize}
		\[
			\text{\textcolor{ForestGreen}{(0.5) resolver.}}
		\]
	\item Recordemos que
		\[
			K(x_0) = \{ d \in \mathbb{R}^n : \nabla h(x_0)^\top d = 0, \nabla f(x_0)^\top d \leq 0\}.
		\]
		En este caso,
		\[
			\nabla h(x, y) = \mqty(\frac{x}{2} \\ 2y), \qquad \nabla f(x, y) = \mqty(2x \\ 2y).
		\]
		\[
			\text{\textcolor{ForestGreen}{(1.0) definición y gradientes involucrados.}}
		\]
		Ahora, evaluamos en los pares de puntos obtenidos:
		\begin{enumerate}
			\item En \((0, \pm 1)\), nos quedan las condiciones
				\begin{align*}
					0 \cdot d_1 \pm 2 \cdot d_2 = 0 &\implies d_2 = 0, \\
					0 \cdot d_1 \pm 2 \cdot d_2 \leq 0 & \implies \pm d_2 \leq 0.
				\end{align*}
				La segunda condición es consecuencia de la primera, de modo que se obtiene
				\[
					K(0, \pm 1) = \qty{\mqty(d_1 \\ 0) : d_1 \in \mathbb{R}}.
				\]
				\[
					\text{\textcolor{ForestGreen}{(0.5) primer grupo.}}
				\]
			\item En \((\pm 2, 0)\), nos quedan las condiciones
				\begin{align*}
					\pm d_1 + 0 \cdot d_2 = 0 &\implies d_1 = 0, \\
					\pm 4d_1 + 0 \cdot d_2 \leq 0 &\implies \pm d_1 \leq 0.
				\end{align*}
				Nuevamente, la segunda condición es consecuencia de la primera, de modo que se obtiene
				\[
					K(\pm 2, 0) = \qty{\mqty(0 \\ d_2) : d_2 \in \mathbb{R}}.
				\]
				\[
					\text{\textcolor{ForestGreen}{(0.5) segundo grupo.}}
				\]
		\end{enumerate}
	\item El Lagrangeano es dado por
		\[
			\mathcal{L}(x, y, \mu) = x^2 + y^2 + \mu \left(\frac{x^2}{4} + y^2 - 1\right).
		\]
		De modo que sus derivadas las obtenemos como sigue:
		\begin{align*}
			\pdv{\mathcal{L}}{x} &= 2x + \frac{\mu x}{2}, \\
			\pdv{\mathcal{L}}{y} &= 2y + 2\mu y, \\
			\pdv[2]{\mathcal{L}}{x} &= 2 + \frac{\mu}{2}, \\
			\pdv[2]{\mathcal{L}}{y} &= 2 + 2\mu, \\
			\frac{\partial^2 \mathcal{L}}{\partial x \partial y} &= 0.
		\end{align*}
		\[
			\text{\textcolor{ForestGreen}{(0.5) cálculo de las derivadas.}}
		\]
		Así, la matriz Hessiana queda
		\[
			\nabla^2_{xy}\mathcal{L}(x, y, \mu) = \mqty(2 + \frac{\mu}{2} & 0 \\ 0 & 2 + 2\mu).
		\]
		En particular,
		\begin{align*}
			\nabla^2_{xy}\mathcal{L}(0, \pm 1, -1) &= \mqty(\frac{3}{2} & 0 \\ 0 & 0), \\
			\nabla^2_{xy}\mathcal{L}(\pm 2, 0, -4) &= \mqty(0 & 0 \\ 0 & -6).
		\end{align*}
		\[
			\text{\textcolor{ForestGreen}{(0.5) conclusión.}}
		\]
	\item Se puede ver que los puntos \((\pm 2, 0)\) no son mínimos locales ya que el cono crítico es no trivial y \(\nabla^2 L\) es semidefinido negativo. Por otro lado, para los puntos \((0, \pm 1)\), sea \(d_1 \in \mathbb{R}\) y vemos que
		\[
			\mqty(d_1 & 0) \mqty(\frac{3}{2} & 0 \\ 0 & 0) \mqty(d_1 \\ 0) = \frac{3}{2} d_1^2 > 0
		\]
		siempre que \(d_1 \neq 0\). Con ello, se concluye que \((0, \pm 1)\) son mínimos locales. \textcolor{magenta}{Nota: No basta decir que la matriz es semidefinida positiva, la desigualdad debe ser estricta.}
		\[
			\text{\textcolor{ForestGreen}{(1.0) clasificar los puntos críticos.}}
		\]

		Ya que a condición de calificación de independencia lineal se cumple en todo punto del conjunto factible y estos puntos \((0, \pm 1)\) tienen igual valor \(1\), corresponden a los mínimos globales. \textcolor{magenta}{Nota: No se puede utilizar la condición de calificación de Slater, el problema no es convexo.}
		\[
			\text{\textcolor{ForestGreen}{(1.0) discutir calificación y concluir.}}
		\]
		\end{enumerate}
	\item 
		\begin{enumerate}
			\item (3pt) Obtenga el minimizador \(x^*(u)\) y la función valor \(v(u)\) asociada al problema
				\begin{align*}
					\min_{x \in \mathbb{R}}\; & x^2 + 1 \\
								  & (x - 2)(x - 4) \leq u,
				\end{align*}
				para cada \(u \in \mathbb{R}\) tal que el problema sea factible.
			\item (2pt) Para \(u \leq 8\), utilice las condiciones de KKT para obtener el multiplicador \(\lambda^*(u)\) asociado a la solución \(x^*(u)\).
			\item (1pt) Verifique que se cumple la igualdad \(v'(u) = -\lambda^*(u)\)
		\end{enumerate}

		\textbf{Solución:}
		\begin{enumerate}
			\item Notemos que, dado \(u \in \mathbb{R}\), el conjunto factible viene dado por la solución de una inecuación cuadrática cuyo vértice es \(3\). Más precisamente, el conjunto factible, cuando es no vacío, viene dado por
				\[
					x \in \qty[3 - \sqrt{1 + u}, 3 + \sqrt{1 + u}].
				\]
				\[
					\text{\textcolor{ForestGreen}{(1.0) describir conjunto factible.}}
				\]
				También note que el mínimo global de la función objetivo es \(x = 0\). De modo que hay dos casos dependiendo de \(u\):
				\begin{itemize}
					\item O \(x^* = 0\) es factible, de modo que es el minimizador,
					\item o el conjunto factible es subconjunto de \(\mathbb{R}_+\) donde la función objetivo es creciente, por lo que el mínimo se alcanzará en el extremo izquierdo del intervalo.
				\end{itemize}
				\[
					\text{\textcolor{ForestGreen}{(1.0) describir dónde se ubica el mínimo.}}
				\]
				La condición para que \(0\) sea factible viene dada por
				\begin{align*}
					& 3 - \sqrt{1 + u} \leq 0 \\
					\implies & \sqrt{1 + u} \geq 3 \\
					\implies & 1 + u \geq 9 \\
					\implies & u \geq 8.
				\end{align*}
				Así, el minimizador viene dado por
				\[
					x^*(u) = \begin{cases}
						0 & u > 8, \\
						3 - \sqrt{1+u} & u \leq 8.
					\end{cases}
				\]
				En consecuencia, la función valor es:
				\[
					v(u)= \begin{cases}
						1 & u > 8, \\
						11 - 6 \sqrt{1 + u} + u & u \leq 8.
					\end{cases}
				\]
				\[
					\text{\textcolor{ForestGreen}{(1.0) concluir.}}
				\]
			\item El sistema de KKT es dado por
				\begin{align*}
					2x + \lambda(2x - 6) &= 0, \\
					\left((x - 2)(x - 4) - u \leq 0\right) &\perp (\lambda \geq 0).
				\end{align*}
				\[
					\text{\textcolor{ForestGreen}{(1.0) plantear sistema.}}
				\]
				\textcolor{ForestGreen}{(Nota: La última condición puede escribirse como las dos desigualdades y \(\lambda \cdot ((x - 2)(x - 4) - u) = 0\))}

				Para \(u \leq 8\), sabemos que \(x = 3 - \sqrt{1 + u}\). De modo que despejamos el multiplicador de la ecuación de KKT, obteniendo
				\[
					\lambda^*(u) = \frac{3}{\sqrt{1 + u}} - 1.
				\]
				\[
					\text{\textcolor{ForestGreen}{(0.8) obtener multiplicador.}}
				\]
				Nótese que esto es efectivamente no negativo para \(u \leq 8\).
				\[
					\text{\textcolor{ForestGreen}{(0.2) abordar positividad.}}
				\]
			\item Derivando directamente la función valor obtenida en a), vemos que
				\[
					v'(u) = 1 - \frac{6}{2 \sqrt{1+u}} = 1 - \frac{3}{\sqrt{1 + u}} = - \lambda^*(u).
				\]
				\[
					\text{\textcolor{ForestGreen}{(1.0) derivación.}}
				\]
		\end{enumerate}
\end{enumerate}

\end{document}
