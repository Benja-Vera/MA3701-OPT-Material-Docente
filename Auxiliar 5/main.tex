\documentclass{article}
\input{imports.tex}
\input{config.tex}

\begin{document}
\noindent \textbf{MA3701 Optimización}\\
\textbf{Profesor:} Alejandro Jofré\\
\textbf{Auxiliar:} Benjamín Vera Vera


\begin{center}
    \Huge{\textbf{Auxiliar 5}}\\
\textit{\large{Método de máximo descenso}}\\
    \normalsize
    \today
\end{center}

\begin{enumerate}
	\item Sea $f: \mathbb{R}^n \to \mathbb{R}$ tal que \(\nabla f\) es \(L\)-Lipschitz. En esta pregunta obtendremos garantías teóricas sobre el ritmo de convergencia del método iterativo del máximo descenso con paso fijo dado por
		\[
			x_{k+1} = x_k - \alpha \nabla f(x_k).
		\]
	en que \(\alpha > 0\) se debe elegir de manera óptima. Para esto, proceda como sigue:
	\begin{enumerate}
		\item Pruebe que:
		\[
	\forall x, y \in \mathbb{R}^n : f(y) \leq f(x) + \nabla f(x)^\top (y - x) + \frac{L}{2} \norm{y - x}^2.
		\]
Concluya que para $d \in \mathbb{R}^n, \alpha \in \mathbb{R}$, se tiene que
$$
	f(x + \alpha d) \leq f(x) + \alpha \nabla f(x)^\top d + \alpha^2 \frac{L}{2} \norm{d}^2.
$$
\item Tomando $d = - \nabla f(x)$, obtenga $\overline{\alpha}$ que minimiza el lado derecho de la desigualdad. Deduzca con ello una cota para $f(x_{k+1})$ en que $x_{k+1} = x_k - \overline{\alpha}\nabla f(x)$.
\item A partir de la cota obtenida anteriormente y suponiendo que $f$ es acotada inferiormente por $\overline{f}$, deduzca una fórmula que permita obtener el número de iteraciones requeridas para obtener $x_k$ con $\norm{\nabla f(x_k)} < \epsilon$.
	\end{enumerate}
\item Considere ahora la función \(f: \mathbb{R}^2 \to \mathbb{R}\) dada por \(f(x) = x_1^2 + x_2^2\).
	\begin{enumerate}
		\item Obtenga la constante de Lipschitz de \(\nabla f\) y deduzca el paso \(\overline{\alpha}\) asociado.
		\item Encuentre \(M\) tal que \(x_{k+1} = Mx_k\). Obtenga con ello una fórmula explícita para \(x_k\) si las iteraciones se inician desde \(x_0 = (1, 1)\).
		\item Calcule la norma del gradiente en cada iteración \(x_k\) y compare con la cota obtenida en la pregunta anterior.
	\end{enumerate}
\end{enumerate}

\end{document}
